\documentclass{article}

\usepackage[utf8]{inputenc}
\usepackage[T1]{fontenc}
\usepackage[francais]{babel}
\usepackage[left=3cm, right=3cm, bottom=2cm, top=2cm]{geometry}
\usepackage{fixltx2e}
\usepackage{fancyhdr}
\usepackage{graphicx}
\usepackage{amsmath}
\usepackage{amssymb}
\usepackage{mathrsfs}
\usepackage{stmaryrd}
\usepackage[dvipsnames]{xcolor}
\usepackage[object=vectorian]{pgfornament}
\usetikzlibrary{shapes.geometric,calc}
\usepackage{algorithm2e}
\newcommand{\sep}{\begin{center}
        \pgfornament[width=0.25\linewidth]{84}
\end{center}}
\newcommand{\seg}[2]{\llbracket #1, #2 \rrbracket}

\MakeRobust{\overrightarrow}

\pagestyle{fancy}
\lhead{Devoir maison 1}
\rhead{Luc Chabassier}

\newcommand{\vc}[1]{\overrightarrow{#1}}
\newcommand{\A}{\mathscr{A}}
\newcommand{\F}{\mathscr{F}}
\newcommand{\B}{\mathscr{B}}
\newcommand{\parts}{\mathscr{P}}
\newcommand{\ntoi}{{n\rightarrow\infty}}
\newcommand{\N}{\mathbb{N}}
\newcommand{\R}{\mathbb{R}}
\newcommand{\Q}{\mathbb{Q}}

\title{Processeur en NetList\\--- Système Digital ---}
\author{Luc Chabassier \and Thibaut Pérami \and Théophile Wallez}

\begin{document}
\maketitle

\section{Introduction}
L'archtecture générale du projet est la suivante. Dans le dossier \texttt{src}
se trouve un code Haskell qui, une fois compilé et exécuté, écrit la NetList.

La NetList est ensuite compilé vers du \emph{c++} à l'aide du compilateur de
NetList situé dans le module git \texttt{NetCompil}. Le fichier doit alors être
compilé avec \texttt{g++} (en mode \emph{c++11}, et avec la \emph{sfml} comme
dépendance).

Le fichier ainsi obtenu peut alors être exécuté. Le contenu de la rom est celui
d'un fichier externe, dont le chemin est passé en ligne de commande. Le contenu
de ce fichier doit être celui d'une suite d'intruction telle que spécifié dans
\texttt{Specification}.

Afin de faciliter la génération de ce fichier, un assembleur est implémenté en
\emph{c++} dans le dossier \texttt{asm}.

Le code assembleur qui contient le code de l'horloge est dans \texttt{clock}.
Le code lui-même n'est pas rédigé en assembleur simple, mais contient des
macros \emph{m4}.

\section{La génération de NetList}
Pour la génération de NetList, on a développé un \emph{DSL} (pour \emph{Domain
Specific Language}) pour écrire la NetList de manière relativement naturelle
en Haskell, en se basant sur une \emph{State monad} qui permet de générer
automatiquement les labels des variables.

Ce système est cependant limité puisque que l'on ne peut pas composer les
intructions : on est obligé de créer autant de variables intermédiaires qu'il
n'y a d'instructions. Mais on bénéficie de la puissance de Haskell, c'est à dire
la récursivité (qui simplifie la création d'\emph{adder} par exemple), la
modularité pour découper le code en fichiers et le rendre plus lisible, et les
opérations usuelles sur les monades.

Le système atteint cependant ses limites lors de l'écriture de circuits trop
intriqués (le système introduit des dépendances entre les variables qui empêche
de juste laisser la paresse de Haskell travailler à notre place). Pour
contourner ce problème, un système permettant de créer des \emph{fausse}
variables qui lors de la générations sont remplacées par des vraies.

\section{Le processeur}
Le processeur est découpé en plusieurs modules :\begin{itemize}
    \item L'\emph{ALU} est le circuit s'occupant des opérations arithmétiques.
    \item Le système de \emph{flags} crée les flags et s'occupe du décodage
        des codes de flags pour retourner la valeur du bon flag.
    \item Le système de registres crée les registres et décodes les code de
        registres avec les instructions d'écriture.
    \item Le système de mémoire s'occupe du décodage des instructions de RAM
        et de leur exécution.
    \item Le \emph{GPU} s'occupe de la communication asynchrone avec le GPU.
    \item Le système de \emph{Test} donne les bonnes instructions à l'ALU et
        interprète son résultat pour déterminer la valeur d'un test.
    \item Le système d'instruction commande les autres systèmes et décodes le
        instructions en ROM.
\end{itemize}

\section{Le simulateur}

\section{L'assembleur}

\section{L'horloge}
L'assembleur lit le timestamp dans la RAM.

Pour convertir en année, mois, jour, ... on fait des soustractions
successives au timestamp.

Pour les années bissextilles, on retient l'année courante modulo 4, 100 et
400

Pour le décompte des secondes, on attends le changement de seconde en
regardant quand les 16 bits de poids faible du timestamp changent.

Pour gérer les années bissextilles, on stocke en RAM le nombre de jours
dans chaque mois.

Lors d'un changement d'année, on met à jour le nombre de jours de février

\end{document}

