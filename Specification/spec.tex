\documentclass[12pt]{article}

\usepackage[utf8]{inputenc} 
\usepackage[T1]{fontenc}      
\usepackage[francais]{babel} 


% maths packages
\usepackage{amsmath}
\usepackage{amsfonts}
\usepackage{amssymb}
\usepackage{asymptote}


\usepackage[bottom=2cm,left=2cm,right=2cm,top=2cm]{geometry}

\title{Spécification langage-machine}

\author{Thibaut PÉRAMI, Théophile WALLEZ, Luc CHABASSIER}


\begin{document}

\maketitle

L'architecture choisi est 16 bit tout les calculs d'entier se font sur 16 bits,
les adresses sont sur 16 bits on ne peut donc controller que $2^{16}$ octets de
RAM et de ROM. Notamment le programme d'entrée ne peu excéder $2^{16}$ octets.

\section{Registres et drapeaux}

\paragraph{Registres :}
\begin{itemize}
\item Registre 0 : \verb!ret!. Ce registre sert à stocker la valeur de retour
  d'un appel de fonction. Il est \textit{caller-saved}.
\item Registres 1 et 2: \verb!Hi! et \verb!Lo!. Ces registres servent pour les
  instruction de multiplication et de division. Il sont \textit{caller-saved}.
\item Registres 3 à 6 : \verb!a0! à \verb!a3!. Ces registres sont utilisé pour
  le passage des arguments. Il sont \textit{caller-saved}.
\item Registre 7 : \verb!sp!. Ce registre contient le pointeur de pile. Il est
  \textit{callee-saved}
\item Registre 8 : \verb!fp!. Ce registre contient le pointeur de frame. Il est
  \textit{callee-saved}
  \item Registre 9 à 15 : \verb!r0! à \verb!r6!. Ces registres sont génériques.
    \verb!r0! à \verb!r2! sont \textit{callee-saved} et \verb!r3! à \verb!r6! sont \textit{caller-saved}
\end{itemize}

\paragraph{Flags :}
\begin{itemize}
\item Flag \verb!Z! : Est vrai si le résultat du dernier calcul est nul
\item Flag \verb!C! : Est vrai si le résultat du dernier calcul a déclanché une
  retenue (n'a pas de sens sur des entier signés)
\item Flag \verb!P! : Est vrai si le résultat est positif i.e si le bit 15
  est nul.
\item Flag \verb!O! : Est vrai si le dernier calcul a déclenché un overflow (n'a
  pas de sens sur des addition/soustraction non signés)
\item Flag \verb!S! : Est vrai si le résultat du dernier calcul vaut 60.
\end{itemize}

\section{Représentation Binaire}

\paragraph{Intructions $O$ :} Ce sont des instruction de type opération. Elle
ont pour forme : 
\begin{center}
  \begin{asy}
    unitsize(0.25cm);
    void square(string lab,int start,int width,bool end = false)
    {
      draw((start,0)--(start +width,0)--(start + width,2)--(start,2)--cycle);
      label(lab,(start +width/2,0 ),N,basealign);
      label ("$_{" +(string)start+ "}$",(start,0),S);
      if(end) label ("$_{" +(string)(start + width)+ "}$",(start + width,0),S);
    }
    square("op",0,4); 
    square("dest",4,4);
    square("src",8,4);
    square("func",12,4,true);
  \end{asy}
\end{center}

Le champs ``op'' spécifie le type d'opération et le champ ``func'' spécifie
l'opération exacte.

\paragraph{Instruction $I$ :} Ce sont des instruction de chargement de valeur
immédiate 8bits.

\begin{center}
  \begin{asy}
    unitsize(0.25cm);
    void square(string lab,int start,int width,bool end = false)
    {
      draw((start,0)--(start +width,0)--(start + width,2)--(start,2)--cycle);
      label(lab,(start +width/2,0 ),N,basealign);
      label ("$_{" +(string)start+ "}$",(start,0),S);
      if(end) label ("$_{" +(string)(start + width)+ "}$",(start + width,0),S);
    }
    square("op",0,4); 
    square("dest",4,4);
    square("imm",8,8,true);
  \end{asy}
\end{center}

On charge la valeur ``Imm'' dans ``dest'' de la façon spécifié dans ``op''.

\paragraph{Instruction $J$ :} Ce sont des instructions de saut.

\begin{center}
  \begin{asy}
    unitsize(0.25cm);
    void square(string lab,int start,int width,bool end = false)
    {
      draw((start,0)--(start +width,0)--(start + width,2)--(start,2)--cycle);
      label(lab,(start +width/2,0 ),N,basealign);
      label ("$_{" +(string)start+ "}$",(start,0),S);
      if(end) label ("$_{" +(string)(start + width)+ "}$",(start + width,0),S);
    }
    square("op",0,4); 
    square("dest",4,4);
    square("src",8,4);
    square("func",12,4);
    square("addr",16,16,true);
  \end{asy}
\end{center}

On fait le test spécifié par ``op'' et ``func'' (qui peut être simplement vrai)
sur ``dest'' et ``src'' et on saute à ``addr'' si vrai.

\section{Opcodes}

\end{document}