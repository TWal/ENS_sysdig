\documentclass[12pt]{article}

\usepackage[utf8]{inputenc} 
\usepackage[T1]{fontenc}      
\usepackage[francais]{babel} 


% maths packages
\usepackage{amsmath}
\usepackage{amsfonts}
\usepackage{amssymb}
\usepackage{asymptote}


\usepackage[bottom=2cm,left=2cm,right=2cm,top=2cm]{geometry}

\title{Spécification langage-machine}

\author{Thibaut PÉRAMI, Théophile WALLEZ, Luc CHABASSIER}


\begin{document}

\maketitle

L'architecture choisie est 16 bit tout les calculs d'entier se font sur 16 bits,
les adresses sont sur 16 bits on ne peut donc contrôler que $2^{16}$ octets de
RAM et de ROM. Notamment le programme d'entrée ne peut excéder $2^{16}$ octets.

\section{Registres et drapeaux}

\paragraph{Registres :}
\begin{itemize}
\item Registre 0 : \verb!ret!. Ce registre sert à stocker la valeur de retour
  d'un appel de fonction. Il est \textit{caller-saved}.
\item Registres 1 et 2: \verb!Hi! et \verb!Lo!. Ces registres servent pour les
  instruction de multiplication et de division. Il sont \textit{caller-saved}.
\item Registres 3 à 6 : \verb!a0! à \verb!a3!. Ces registres sont utilisé pour
  le passage des arguments. Il sont \textit{caller-saved}.
\item Registre 7 : \verb!sp!. Ce registre contient le pointeur de pile. Il est
  \textit{callee-saved}
\item Registre 8 : \verb!fp!. Ce registre contient le pointeur de frame. Il est
  \textit{callee-saved}
  \item Registre 9 à 15 : \verb!r0! à \verb!r6!. Ces registres sont génériques.
    \verb!r0! à \verb!r2! sont \textit{callee-saved} et \verb!r3! à \verb!r6! sont \textit{caller-saved}
\end{itemize}

\paragraph{Drapeaux :}
\begin{itemize}
\item Flag \verb!Z! : Est vrai si le résultat du dernier calcul est nul
\item Flag \verb!C! : Est vrai si le résultat du dernier calcul a déclanché une
  retenue (n'a pas de sens sur des entier signés)
\item Flag \verb!P! : Est vrai si le résultat est positif i.e si le bit 15
  est nul.
\item Flag \verb!O! : Est vrai si le dernier calcul a déclenché un overflow (n'a
  pas de sens sur des addition/soustraction non signés)
\item Flag \verb!S! : Est vrai si le résultat du dernier calcul vaut 60.
\end{itemize}

\section{Représentation Binaire}

Le champs ``op'' spécifie le type d'opération et le champ ``func'' spécifie
l'opération exacte.
\paragraph{Intructions $S$ :} Ce sont des instructions courtes. Elles
ont pour forme : 
\begin{center}
  \begin{asy}
    unitsize(0.25cm);
    void square(string lab,int start,int width,bool end = false)
    {
      draw((start,0)--(start +width,0)--(start + width,2)--(start,2)--cycle);
      label(lab,(start +width/2,0 ),N,basealign);
      label ("$_{" +(string)start+ "}$",(start,0),S);
      if(end) label ("$_{" +(string)(start + width)+ "}$",(start + width,0),S);
    }
    square("op",0,4); 
    square("dest",4,4);
    square("src",8,4);
    square("func",12,4,true);
  \end{asy}
\end{center}


\paragraph{Instruction $L$ :} Ce sont des instructions longues qui ont besoin
d'une valeur sur 16 bits (addresse ou constante).

\begin{center}
  \begin{asy}
    unitsize(0.25cm);
    void square(string lab,int start,int width,bool end = false)
    {
      draw((start,0)--(start +width,0)--(start + width,2)--(start,2)--cycle);
      label(lab,(start +width/2,0 ),N,basealign);
      label ("$_{" +(string)start+ "}$",(start,0),S);
      if(end) label ("$_{" +(string)(start + width)+ "}$",(start + width,0),S);
    }
    square("op",0,4); 
    square("dest",4,4);
    square("src",8,4);
    square("func",12,4);
    square("val",16,16,true);
  \end{asy}
\end{center}


\section{Opcodes}

Les codes sont représentés sur 4 bits. Les instructions courtes commencent par un
0 et les longues par un 1.

\paragraph{Intructions courtes :}
\begin{itemize}
\item Code 0 : Intructions Binaire : Contient un code fonction de type
  \verb!bin! qui décrit un opération binaire, cette opération est faite entre
  \verb!src! et \verb!dest! et le résultat est écrit dans \verb!dest!
\item Code 1 : Intructions Unaires : Contient un code fonction de type \verb!un!
  qui décrit une opération unaire faite sur le registre \verb!dest!. le champs \verb!src!
  peu servir de valeur immédiate sur 4 bits.
\item Code 2 : Instruction Mémoire : Contient un code fonction de type
  \verb!mem! qui décrit un opération mémoire.
\item Code 4 : Saut simple sur drapeaux  : Contient un code fonction de type \verb!flag!
  et saute l'instruction suivante si le code de drapeau correspond à la valeur
  \verb!true!.
\item Code 5 : Saut simple sur test : Contient un code fonction de type
  \verb!test!, effectue ce test sur \verb!src! et \verb!dest! et saute
  l'instruction suivante si vrai.
\end{itemize}

\paragraph{Instructions longues :}
\begin{itemize}
\item Code 8 : Saut long sur drapeaux : Contient un code fonction de type \verb!flag!
  et saute sur l'instruction pointée par \verb!val! si le code de drapeaux
  correspond à \verb!true!.
\item Code 9 : Saut long sur test : Contient un code fonction de type
  \verb!test!, effectue ce test sur \verb!src! et \verb!dest! et saute
  sur l'instruction pointée par \verb!val! si vrai.
\item Code 10 : Saut inconditionel (\verb!jmp!) : Saute à l'adresse pointé par \verb!val!
\item Code 11 : Appel (\verb!call!) : empile la position courante et saute vers l'adresse
  \verb!val!
\item Code 12 : Instruction mémoire à offset : Contient un code fonction de type
  \verb!mem! mais l'adresse mémoire utilisée (\verb!src! ou \verb!dest! en
  fonction du code fonction) de voit ajouté le contenu de \verb!val!
\item Code 13 : Valeur Immédiate (\verb!limm!) : Charge la valeur de \verb!val! dans \verb!dest! 
\item Code 15 : GPU (\verb!gpu!) : Envoie une instruction donnée au GPU. Le
  champs Val doit avoir ses 2 bits de poids faibles à 0 et contient instr sur
  ses bits de 2 à 7 et modif sur ses bits de 8 à 15 (voir section sur le gpu.)
  Le champs \verb!src! contient le registre qui remplira le champs \verb!data!
  de l'instruction GPU.
\end{itemize}

\section{Codes fonctions}

Sauf mention contraire, une instruction peut potentiellement modifier tous les
drapeaux.

L'écriture mathématique désigne la valeur : $dest$ désigne la valeur de \verb!dest!.

$a$ et $b$ désigne le \og et \fg{} bit à bit de a et b. $\neg a$ désigne la
négation bit à bit de $a$.

Pour deux registre \verb!a! et \verb!b!, \verb!a:b! désigne la registre à 32 bit
concaténation de \verb!a! et \verb!b! (\verb!a! désigne les bits de poids fort).


\paragraph{Code binaires :} Codes de type \verb!bin!.
\begin{itemize}
\item Code 0 : \verb!mov! : écrit $src$ dans \verb!dest!.
\item Code 1 : \verb!movs! (mov silencieux) : écrit $src$ dans \verb!dest!
  sans modifier les drapeaux.
\item Code 4 : \verb!lsr! (shift logique à droite) : fait le shift logique
  (insère des 0) à droite de \verb!dest! de \verb!src! bits, écrit dans \verb!dest!.
\item Code 5 : \verb!lsl! (shift logique à gauche) : fait le shift logique à
  gauche de \verb!dest! de \verb!src! bits, écrit dans \verb!dest!.
\item Code 6 : \verb!asr! (shift arithmétique à droite) : fait le shift arithmétique à
  droite (recopie du bit de signe) de \verb!dest! de \verb!src! bits, écrit dans \verb!dest!.
\item Code 8 : \verb!add! : écrit $src + dest$ dans \verb!dest!.
\item Code 9 : \verb!addc! : écrit $src + dest + c$ dans \verb!dest! ($c$
  désigne la valeur du drapeau \verb!C!).
\item Code 10 : \verb!sub! : écrit $dest - src$ dans \verb!dest!
\item Code 11 : \verb!subc! : écrit $dest - src -c $ dans \verb!dest!.
\item Code 12 : \verb!and! : écrit $dest$ et $src$ dans \verb!dest!.
\item Code 13 : \verb!or! : écrit $dest$ ou $src$ dans \verb!dest!.
\item Code 14 : \verb!xor! : écrit $dest \oplus src$ dans \verb!dest!.
\item Code 15 : \verb!nand! : écrit $\neg(dest$ et $src)$ dans \verb!dest!.

\end{itemize}


\paragraph{Codes Unaires :} Codes de type \verb!un!.i
\begin{itemize}
\item Code 0 : \verb!not! : écrit $\neg dest$ dans \verb!dest!.
\item Code 1 : \verb!decr! : écrit $dest-1$ dans \verb!dest!.
\item Code 2 : \verb!incr! : écrit $dest +1$ dans \verb!dest!.
\item Code 3 : \verb!neg! : écrit $ - dest$ dans \verb!dest!.
\item Code 4 : \verb!lsri! Fait un shift logique droite avec la valeur immédiate
  sur 4 bits stocké dans \verb!src!.
\item Code 5 : \verb!lsli! Fait un shift logique gauche avec la valeur immédiate
  sur 4 bits stocké dans \verb!src!.
\item Code 6 : \verb!asri! Fait un shift arithmétique droite avec la valeur immédiate
  sur 4 bits stocké dans \verb!src!.

\item Code 8 : \verb!mulu! : écrit le résultat de la multiplication non signée de
  \verb!Lo! et \verb!dest! dans \verb!Hi:Lo! 
\item Code 9 : \verb!muls! : écrit le résultat de la multiplication signée de
  \verb!Lo! et \verb!dest! dans \verb!Hi:Lo! 
\item Code 10 : \verb!divu! : calcule la division non signée de \verb!Hi:Lo! par
  \verb!dest!, met le quotient dans \verb!Hi! et le reste dans \verb!Lo!. Si
  \verb!dest! était nul, le drapeau \verb!O! est levé et la valeur de
  \verb!Hi! et \verb!Lo! n'est pas définie.
  Les drapeaux \verb!P! et \verb!S! sont mis en fonction du quotient, le
  drapeau \verb!Z! est mis en fonction du reste.
\item Code 11 : \verb!divs! La même chose mais avec une division signée.
\item Code 14 : \verb!div10! Racourci de \verb!divu! avec un registre de valeur 10.
\item Code 15 : \verb!div10i! Racourci de \verb!divi! avec un registre de valeur 10.
\end{itemize}

\paragraph{Instruction Mémoires :} Codes de type \verb!mem!.
\begin{itemize}
\item Code 0 : \verb!rdw! : lit le mot de 16bits à l'adresse $src$ et l'écrit
  dans \verb!dest!.
\item Code 1 : \verb!rdbu! : lit le mot de 8 bits à l'adresse $src$ et l'écrit
  dans \verb!dest! en complétant les bit de poids fort avec des zéros.
\item Code 2 : \verb!rdbi! : lit le mot de 8 bits à l'adresse $src$ et l'écrit
  dans \verb!dest! en recopiant le bit de signe sur les bit de poids fort.
\item Code 4: \verb!wrw! : écrit le mot de 16 bits $src$ à l'adresse $dest$.
\item Code 5 : \verb!wrl! : écrit le mot de 8 bits des bits de poids faible de
  \verb!src! à l'adresse $dest$.
\item Code 6 : \verb!wrh! : écrit le mot de 8 bits des bits de poids fort de
  \verb!src! à l'adresse $dest$.
\item Code 8 : \verb!push! : écrit le mot de 16 bits $dest$ à l'adresse $sp$ et
  décrémente  \verb!sp! de 2.
\item Code 9 : \verb!pop! : incrémente \verb!sp! de 2 et lit le mot à l'adresse $sp$
  et le place dans \verb!dest!.
\item Code 10 : \verb!ret! : incrémente \verb!sp! de 2 et fait un jump vers l'adresse
  pointée par $sp$
\end{itemize}


\paragraph{Code de Drapeaux :} Code de type \verb!flag!. Si c'est le nom d'un
drapeau en minuscule, cela vaut vrai si le drapeau est levé, si c'est précéde
d'un \verb!n!, alors c'est l'inverse.
\begin{itemize}
\item Code 0 : \verb!z!. 
\item Code 1 : \verb!nz!.
\item Code 2 : \verb!c!
\item Code 3 : \verb!nc!
\item Code 4 : \verb!p!
\item Code 5 : \verb!np!
\item Code 6 : \verb!o!
\item Code 7 : \verb!no!
\item Code 8 : \verb!s!
\item Code 9 : \verb!ns!
\end{itemize}


\paragraph{Code de tests :} Code de type \verb!test!. On donne à chaque fois
l'expression logique calculée.
\begin{itemize}
\item Code 0 : \verb!eq! : $dest = src$
\item Code 1 : \verb!neq! : $dest \neq src$
\item Code 2 : \verb!g! : $dest > src$ (non signé)
\item Code 3 : \verb!ge! : $dest \ge src$ (non signé)
\item Code 4 : \verb!l! : $dest < src$ (non signé)
\item Code 5 : \verb!le! : $dest \le src$ (non signé)
\item Code 6 : \verb!gi! : $dest > src$ (signé)
\item Code 7 : \verb!gei! : $dest \ge src$ (signé)
\item Code 8 : \verb!li! : $dest < src$ (signé)
\item Code 9 : \verb!lei! : $dest \le src$ (signé)
\item Code 10 : \verb!andz! : $(dest$ et $src) = 0$
\item Code 11 : \verb!nandz! : $\neg(dest$ et $src) = 0$
\item Code 12 : \verb!xorz! : $(dest \oplus src) = 0$
\item Code 13 : \verb!nxorz! : $\neg(dest \oplus src) = 0$
\item Code 14 : \verb!eq60! : $dest = 60$
\item Code 15 : \verb!neq60! : $dest \neq 60$
\end{itemize}

\section{Syntaxe de l'assembleur}

\paragraph{Positionement :} Chaque instruction attend un registe ou une valeur à
une position determinée, il est donc inutile de préciser qu'on est un registre,
un label ou un valeur immédiate à l'aide d'un symbole comme \%.


\paragraph{Instruction standard :} On note la source à gauche et la destination
à droite : dans \verb!mov a b!, on a \verb!dest = b! et \verb!src = a!

\paragraph{Labels :} On rajoute \verb!label :! dans le code pour donner le nom \og
label \fg{}  à l'adresse où l'on est.

\paragraph{Sauts :} Pour créer le nom d'un instruction de saut, on concatène le
type de saut avec le nom du code de fonction. Les préfixes sont \verb!js! pour
le saut court et \verb!j! pour les longs. L'instruction \verb!jsz! saute
l'instruction suivante si le drapeau \verb!Z! est levé.

L'instruction \verb!jeq a b pos! saute au label \verb!pos! si $a = b$.

\paragraph{Mémoire :} La différence entre les instruction mémoire avec ou sans
offset se fait uniquement par la présence d'un entier derrière : \verb!wrl a b! et
\verb!wrl a b 0! auront le même comportement (écrire $a$ dans l'adresse $b$).

\paragraph{Valeur Immédiate :} On écrira \verb!limm 42 a! pour charger la valeur
42 dans \verb!a!.

\paragraph{Instruction GPU :} On écrira \verb!gpu wmi r2! pour dire au GPU de
mettre la valeur de \verb!r2! dans le champs des minutes.

\paragraph{Commentaire :} On mettra un \# devant une ligne pour la commenter.

\section{Mapping RAM}

\paragraph{Zone spéciales :}
\begin{itemize}
\item Adresse 0-3 : Code d'erreur : si cette valeur deviens non nulle, le
  processeur s'arrète immédiatement et affiche ce code.
\item Address 4-7 : Carte graphique : ce qui est placé dans cette adresse sera
  donné à la carte graphique.
\item Address 8-11 : Temps : entier 32-bit contenant le temps écoulé depuis le
  1er janvier 1970.
\end{itemize}

\section{Interface avec carte Graphique}

\paragraph{Forme d'instruction :}

\begin{center}
  \begin{asy}
    unitsize(0.25cm);
    void square(string lab,int start,int width,bool end = false)
    {
      draw((start,0)--(start +width,0)--(start + width,2)--(start,2)--cycle);
      label(lab,(start +width/2,0 ),N,basealign);
      label ("$_{" +(string)start+ "}$",(start,0),S);
      if(end) label ("$_{" +(string)(start + width)+ "}$",(start + width,0),S);
    }
    square("c",0,2); 
    square("instr",2,6);
    square("modif",8,8);
    square("data",16,16,true);
  \end{asy}
\end{center}

\paragraph{Protocole :} le code $c$ décrit l'état de la communication :
\begin{itemize}
\item Code 0 : Aucune comminication : on peut écrire
\item Code 1 : CPU entrain d'écrire.
\item Code 2 : CPU fini d'écrire, message pas encore lu par GPU
\end{itemize}

\paragraph{Éxecution du protocole :}
\begin{itemize}
\item Le CPU attend le code 0.
\item Le CPU met le code à 1.
\item Le CPU écrit le contenu.
\item Le CPU met le code à 2.
\item Dès qu'il voit un 2, le GPU lit.
\item Le GPU met le code à 0 à la fin de sa lecture.
\end{itemize}

\paragraph{Instructions :}
\begin{itemize}
\item Code 0 \verb!sseg! : Passe en mode afficheur 7-segment (où cas où on en mette un
  autre).
\item Code 1 - 7 : écrit la nouvelle valeur de cette unité dans l'afficheur (1 :
  \verb!ws!, secondes, 2 : \verb!wmi!, minutes, 3 : \verb!wh!, heures , 4 : \verb!wwd! jour de la
  semaine , 5 : \verb!wd! jour du mois , 6 : \verb!wmo! : mois , 7  : \verb!wy! : année) la valeur est lue sur les 16-bits de poids fort de data (bit 16 à 31
  de l'instruction).
\end{itemize}

\end{document}